\documentclass[11pt]{article}
\usepackage{fullpage}
\usepackage{url}
\usepackage{color}
\usepackage{amsmath, amssymb}
\usepackage{hyperref}
\usepackage{graphicx}
\textheight=8.85in
\usepackage{matlab-prettifier}
\pagestyle{myheadings}
\usepackage{subcaption}

\setlength{\tabcolsep}{0in}
\begin{document}


\thispagestyle {empty}

\newcommand{\lsp}[1]{\large\renewcommand{\baselinestretch}{#1}\normalsize}
\newcommand{\hsp}{\hspace{.2in}}
\newcommand{\comment}[1]{}
\newtheorem{thm}{Theorem}[section]
\newtheorem{lem}{Lemma}[section]
\newtheorem{cor}{Corollary}[section]
\newtheorem{prop}{Proposition}[section]
\newtheorem{problem}{Problem}[section]

\newcommand{\R}{{\rm\hbox{I\kern-.15em R}}}
\newcommand{\IR}{{\rm\hbox{I\kern-.15em R}}}
\newcommand{\IN}{{\rm\hbox{I\kern-.15em N}}}
\newcommand{\IZ}{{\sf\hbox{Z\kern-.40em Z}}}
\newcommand{\IS}{{\rm\hbox{S\kern-.45em S}}}
\newcommand{\Real}{I\!\!R}


\newcommand{\linesep}{\vspace{.2cm}\hrule\vspace{0.2cm}}
\newcommand{\categorysep}{\vspace{0.5cm}}
\newcommand{\entrysep}{\vspace{0cm}}

\newcommand{\category}[1]{\categorysep
                  \noindent {\bf \large #1}
              \linesep}

\pagestyle{empty}

\begin{center}
{\large \textbf{CSE 847 (Spring 2023): Machine Learning--- Homework 5\\
 Instructor: Jiayu Zhou \\
  Balija Santoshkumar}} \\
\href{mailto:balijasa@msu.edu}{balijasa@msu.edu}
\end{center}
\section{ Clustering: K-means}

\begin{enumerate}
\item  Elaborate the relationship between $k$-means and spectral relaxation of $k$-means. Is it possible that we obtain exact $k$-means solution using spectral relaxed $k$-means?
\item Implementation of $k$-means. Submit all the source code to D2L along with a short report on your observation.
\begin{itemize}
	\item Implement the $k$-means in MATLAB using the alternating procedure introduced in the class (you will not get the credit if you use the build-in kmeans function in MATLAB).
\item  Implement the spectral relaxation of $k$-means. Create a random dataset and compare the $k$-means and spectral relaxed $k$-means.
\end{itemize}
\end{enumerate}

\textbf{Sol:}

We assume that we have $n$ data points $\left\{x_i\right\}_{i=1}^n \in \mathbb{R}^m$, which we organize as columns in a matrix
$$
X=\left[x_1, x_2, \cdots, x_n\right] \in \mathbb{R}^{m \times n}
$$


The objective of K-means is reduce sum squared error(SSE)

\begin{equation}
	q_j=\sum_{v \in \pi_j}\left\|x_v-c_j\right\|^2
\end{equation}
where $c_j$ is the cluster center of the corresponding cluster
Let $e$ be the vector of all ones with appropriate length. It is easy to see that $c_j=X_j e / n_j$, where $X_j$ is the data matrix of the $j$-th cluster.

SSE can transformed into
\begin{equation}
	q_j=\sum_{j=1}^k\left(\operatorname{trace}\left(X_j^T X_j\right)-\frac{e^T}{\sqrt{n_j}} X_j^T X_j \frac{e}{\sqrt{n_j}}\right)
\end{equation}

Define the $n$-by- $k$ orthogonal matrix $Y$ as follows
$$
Y=\left(\begin{array}{cccc}
	e / \sqrt{n_1} & & \\
	& e / \sqrt{n_2} & \\
	& & \vdots & \\
	& & & e / \sqrt{n_k}
\end{array}\right)
$$
Then
$$
Q(\Pi)=\operatorname{trace}\left(X^T X\right)-\operatorname{trace}\left(Y^T X^T X Y\right) .
$$
The $k$-means objective, minimization of $Q(\Pi)$, is equivalent to the maximization of trace $\left(Y^T X^T X Y\right)$ with $Y$ .

In  spectral
relaxation of k-means is instead of using this specific expression for Y , it is
possible to use any arbitrary orthogonal matrix for Y . This leads to the relaxed maximization problem
\begin{equation}
	\max _{Y^T Y=I_k} \operatorname{trace}\left(Y^T X^T X Y\right)
\end{equation}
The first k vectors in the left singular matrix of X can produce the Y   that maximizes this expression in Eq.3
we can see that both k-means and spectral k-means are trying to minimize the same error function, but spectral k-means is  first trying to project the dataset into a lower dimensional space which makes it easier to capture the complex clustering structures.

The spectral-relaxed k-means become completely equivalent to k-means when the expression for Y becomes
equal to the matrix mentioned in Eq. 2

\\

I have used fisheriris data set for K-means

\\
\textbf{Regular K-means}

After implementation we see SSE as

\begin{verbatim}
SSE for k=3: 26.893170
SSE for k=4: 26.067859
SSE for k=5: 14.691709
SSE for k=6: 11.985356
SSE for k=7: 8.231259
SSE for k=8: 6.774484
SSE for k=9: 6.016505
SSE for k=10: 5.094189
\end{verbatim} 


\begin{figure}[htbp]
	\centering
	\begin{subfigure}[b]{0.3\textwidth}
		\includegraphics[width=\textwidth]{./Results/KMeans/Cluster_Arrangement_3.png}
		\caption{No of Clusters 3}
		\label{fig:cluster3}
	\end{subfigure}
	\begin{subfigure}[b]{0.3\textwidth}
		\includegraphics[width=\textwidth]{./Results/KMeans/Cluster_Arrangement_4.png}
		\caption{No of Clusters 4}
		\label{fig:cluster4}
	\end{subfigure}
	\begin{subfigure}[b]{0.3\textwidth}
		\includegraphics[width=\textwidth]{./Results/KMeans/Cluster_Arrangement_5.png}
		\caption{No of Clusters 5}
		\label{fig:cluster5}
	\end{subfigure}
	
	\begin{subfigure}[b]{0.3\textwidth}
		\includegraphics[width=\textwidth]{./Results/KMeans/Cluster_Arrangement_6.png}
		\caption{No of Clusters 6}
		\label{fig:cluster6}
	\end{subfigure}
	\begin{subfigure}[b]{0.3\textwidth}
		\includegraphics[width=\textwidth]{./Results/KMeans/Cluster_Arrangement_7.png}
		\caption{No of Clusters7}
		\label{fig:cluster7}
	\end{subfigure}
	\begin{subfigure}[b]{0.3\textwidth}
		\includegraphics[width=\textwidth]{./Results/KMeans/Cluster_Arrangement_8.png}
		\caption{No of Clusters 8}
		\label{fig:cluster8}
	\end{subfigure}
	
	\begin{subfigure}[b]{0.3\textwidth}
		\includegraphics[width=\textwidth]{./Results/KMeans/Cluster_Arrangement_9.png}
		\caption{No of Clusters 9}
		\label{fig:cluster9}
	\end{subfigure}
	\begin{subfigure}[b]{0.3\textwidth}
		\includegraphics[width=\textwidth]{./Results/KMeans/Cluster_Arrangement_10.png}
		\caption{No of Clusters 10}
		\label{fig:cluster10}
	\end{subfigure}
	\caption{Cluster assignments with K-means}
	\label{fig:clusterassignments}
\end{figure}
\begin{figure}[htbp]
	\centering
	\includegraphics[width=0.6\textwidth]{./Results/KMeans/SSE_Convergence.png}
	\caption{Convergence of SSE with Clusters variation in KMeans}
	\label{fig:SSEconvergence}
\end{figure}

\textbf{ spectral relaxation of $k$-means}


After implementation we see Final SSE as

\begin{verbatim}
SSE for k=3: 145.287135
SSE for k=4: 107.839511
SSE for k=5: 127.624815
SSE for k=6: 82.031575
SSE for k=7: 106.387628
SSE for k=8: 70.359438
SSE for k=9: 69.009363
SSE for k=10: 43.252894
\end{verbatim} 

\begin{figure}[htbp]
	\centering
	\begin{subfigure}[b]{0.3\textwidth}
		\includegraphics[width=\textwidth]{./Results/KMeans/Cluster_Arrangement_S3.png}
		\caption{No of Clusters 3}
		\label{fig:cluster3}
	\end{subfigure}
	\begin{subfigure}[b]{0.3\textwidth}
		\includegraphics[width=\textwidth]{./Results/KMeans/Cluster_Arrangement_S4.png}
		\caption{No of Clusters 4}
		\label{fig:cluster4}
	\end{subfigure}
	\begin{subfigure}[b]{0.3\textwidth}
		\includegraphics[width=\textwidth]{./Results/KMeans/Cluster_Arrangement_S5.png}
		\caption{No of Clusters 5}
		\label{fig:cluster5}
	\end{subfigure}
	
	\begin{subfigure}[b]{0.3\textwidth}
		\includegraphics[width=\textwidth]{./Results/KMeans/Cluster_Arrangement_S6.png}
		\caption{No of Clusters 6}
		\label{fig:cluster6}
	\end{subfigure}
	\begin{subfigure}[b]{0.3\textwidth}
		\includegraphics[width=\textwidth]{./Results/KMeans/Cluster_Arrangement_S7.png}
		\caption{No of Clusters7}
		\label{fig:cluster7}
	\end{subfigure}
	\begin{subfigure}[b]{0.3\textwidth}
		\includegraphics[width=\textwidth]{./Results/KMeans/Cluster_Arrangement_S8.png}
		\caption{No of Clusters 8}
		\label{fig:cluster8}
	\end{subfigure}
	
	\begin{subfigure}[b]{0.3\textwidth}
		\includegraphics[width=\textwidth]{./Results/KMeans/Cluster_Arrangement_S9.png}
		\caption{No of Clusters 9}
		\label{fig:cluster9}
	\end{subfigure}
	\begin{subfigure}[b]{0.3\textwidth}
		\includegraphics[width=\textwidth]{./Results/KMeans/Cluster_Arrangement_S10.png}
		\caption{No of Clusters 10}
		\label{fig:cluster10}
	\end{subfigure}
	\caption{Cluster assignments with Spectral relaxation K-means}
	\label{fig:clusterassignments}
\end{figure}
\begin{figure}[htbp]
	\centering
	\includegraphics[width=0.6\textwidth]{./Results/KMeans/SSE_Convergence_S.png}
	\caption{Convergence of SSE with Clusters variation in Spectral relaxation KMeans}
	\label{fig:SSEconvergence}
\end{figure}
\clearpage

\textbf{MATLAB Code}

\begin{lstlisting}[style=Matlab-editor]
	clear all
	clc
	close all
	load fisheriris
	data = meas
	data = data(:,3:4);
	num_clusters = [3:10];
	SSE = zeros(1, size(num_clusters,2));
	for i=1:size(num_clusters,2)
	[cluster_assignments, cluster_centers] = kmeans_cluster(data, num_clusters(i),1);
	SSE(i) = compute_SSE(data, cluster_assignments);
	show_plot(data, cluster_assignments);
	saveas(gcf, strcat('Results/KMeans/Cluster_Arrangement_S',int2str(num_clusters(i)),'.png'));
	end
	fig = figure; plot(num_clusters, SSE); xlabel('K'); ylabel('SSE'); title('Variation of SSE for different values of K');
	saveas(gcf, strcat('Results/KMeans/SSE_Convergence_S.png'));
	
	
	function [cluster_assignments, cluster_centers] = kmeans_cluster(raw_data, k, spectral)
	%     Code to perform k-means clustering
	%     INPUTS:
	%         data     =  (n * m) matrix where n is the number of samples and m
	%                     is the number of features
	%         k        =  integer number of intended clusters
	%         spectral =  boolean value representing spectral k-means if true,
	%                     else standard k-means
	% 
	%     OUTPUTS:
	%         cluster_assignments = labels assigned to the each samples in
	%                               [1,k]
	%         cluster_centers     = final clsuter centers found in the process
	%                               of k-means
	
	% Assigning default values
	if nargin < 3
	if ~exist('spectral')
	spectral=false;
	end
	end
	
	if spectral
	% for spectral relaxation, map the data samples to k-dimensional
	% feature space
	[U, ~, ~] = svd(raw_data);
	projection = U(:, 1:k);
	rand_mat = rand(k,k);
	orth_mat = orth(rand_mat);
	data = projection * orth_mat;
	else
	data = raw_data;
	end
	
	[num_samples, ~] = size(data);
	cluster_assignments = zeros(num_samples, 1);
	temp = randperm(num_samples);
	cluster_center_idx = temp(1:k);
	cluster_centers = data(cluster_center_idx, :);
	change = inf;
	count_iter = 0;
	
	while(change ~= 0)
	% change represents the number cluster assignments that got changed
	% in the current iteration
	count_iter = count_iter+1;
	prev_assignments = cluster_assignments;
	
	for cur_idx=1:num_samples
	min_dist = inf;
	min_idx = -1;
	
	% for each sample, find the cluster center which is at min
	% distance
	for cluster_idx = 1:k
	cur_dist = norm(data(cur_idx,:) - cluster_centers(cluster_idx,:));
	if(cur_dist < min_dist)
	min_dist = cur_dist;
	min_idx = cluster_idx;
	end
	end
	cluster_assignments(cur_idx,1) = min_idx;
	end
	
	for cluster_idx = 1:k
	% get the mean of each cluster
	cluster_centers(cluster_idx,:) = mean(data(cluster_assignments == cluster_idx,:));
	end
	
	change = sum(prev_assignments ~= cluster_assignments);
	%         fprintf('Number of changes in iter %d: %d\n', count_iter, change);
	%         show_plot(raw_data, cluster_assignments);   % plot the clusters
	end
	
	SSE = compute_SSE(raw_data, cluster_assignments);
	fprintf('SSE for k=%d: %f\n', k, SSE);
	
	end
	
	
	function [SSE] = compute_SSE(data, cluster_assignments)
	% Function to compute Sum of Squared Error
	% INPUTS:
	%   data = the dataset used for clustering
	%   cluster_assignments = labels for each sample in the data
	%   cluster_centers = the centers found for each cluster
	% 
	% OUTPUT:
	%   SSE = final sum of squared errors for the cluster config.
	
	num_clusters = size(unique(cluster_assignments),1);
	SSE = 0;
	for cluster_no = 1:num_clusters
	cluster_center = mean(data(cluster_assignments == cluster_no,:));
	SSE = SSE + norm(data(cluster_assignments==cluster_no,:)-cluster_center)^2;
	end
	end
	
	function [] = show_plot(data, labels)
	%   Function to plot the cluster config.
	%   INPUTS:
	%       data = dataset used for clustering
	%       labels = the cluster label assigned to each sample
	% 
	%   OUTPUT:
	%       A plot representing the cluster config.
	
	k = size(unique(labels),1);
	[~, num_features] = size(data);
	
	
	if num_features>2
	pcs = pca(data);
	reduced_data = data * pcs(:, 1:2);
	else
	reduced_data = data;
	end
	
	figure;
	hold on;
	gscatter(reduced_data(:,1),reduced_data(:,2),labels);
	title(strcat('KMeans Clustering Arrangement for K=', int2str(k)));
	hold off;
	pause(2);
	end
	
	
	
 \end{lstlisting}
\clearpage

\section{Principle Component Analysis}

\begin{enumerate}
	\item  Suppose we have the following data points in $2 \mathrm{~d}$ space $(0,0),(-1,2),(-3,6),(1,-2),(3,-6)$.
	\begin{itemize}
\item Draw them on a 2-d plot, each data point being a dot.
\item  What is the first principle component? Given 1-2 sentences justification. You do not need to run MATLAB to get the answer.
\item  What is the second principle component? Given 1-2 sentences justification. You do not need to run MATLAB to get the answer.
\end{itemize}

\item  Experiment: We apply data pre-processing techniques to a collection of handwritten digit images from the USPS dataset (data in MATLAB format: USPS.mat) ${ }^1$. You can load the whole dataset into MATLAB by load USPS.mat. The matrix $A$ contains all the images of size 16 by 16. Each of the 3000 rows in $A$ corresponds to the image of one handwritten digit (between 0 and 9). To visualize a particular image, such as the second one, first you need to convert the vector representation of the image to the matrix representation by $A 2=$ reshape $(A(2,:), 16,16)$, and then use imshow $\left(A 2^{\prime}\right)$ for visualization.

Implement Principal Component Analysis (PCA) using SVD and apply to the data using $p=10,50,100,200$ principal components. Reconstruct images using the selected principal components from part 1.

\begin{itemize}
\item Show the source code links for parts 1 and 2 to your github account.
\item  The total reconstruction error for $p=10,50,100,200$.
\item  A subset (the first two) of the reconstructed images for $p=10,50,100,200$.
Note: The USPS dataset is available at http://www. csie.ntu.edu.tw/ cjlin/libsvmtools/ datasets/multiclass.html\#usps. The image size is 16 by 16 , thus the data dimensionality of the original dataset is 256 . We used a subset of 3000 images in this homework.
\end{itemize}
\end{enumerate}

\textbf{Sol:}

\end{document}
